\chapter{Méthodologie}
\label{chap:methodo}

\begin{summarybox}{Objectifs}
    Ce chapitre présente les méthodes utilisées, en mettant l'accent sur la mise en forme des équations mathématiques complexes et du code.
\end{summarybox}

\section{Modélisation Mathématique}

Nous utilisons des équations pour décrire le système. Grâce au package \texttt{mathtools}, nous pouvons définir des symboles proprement.

\subsection{Équations de base}

L'équation d'Einstein est donnée par :
\begin{equation}
    E = mc^2
    \label{eq:einstein}
\end{equation}

\subsection{Systèmes d'équations}

Voici un exemple d'alignement d'équations avec \texttt{align} :
\begin{align}
    f(x) &= x^2 + 2x + 1 \\
         &= (x+1)^2
\end{align}

\subsection{Définitions et Cas}

L'utilisation de \texttt{cases} est améliorée par \texttt{mathtools} (via \texttt{dcases} pour l'affichage display) :
\begin{equation}
    u(x) = 
    \begin{dcases}
        0 & \text{si } x < 0 \\
        1 & \text{si } x \geq 0
    \end{dcases}
\end{equation}

\section{Implémentation (Code)}

Voici comment nous implémentons la fonction ci-dessus en Python.

\begin{infobox}{Information}
    Le code ci-dessous utilise la bibliothèque standard.
\end{infobox}

\begin{listing}[H]
\begin{minted}{python}
def u(x):
    """Fonction échelon."""
    if x < 0:
        return 0
    else:
        return 1
\end{minted}
\caption{Implémentation de la fonction échelon}
\label{lst:echelon}
\end{listing}

\begin{alertbox}{Attention}
    Assurez-vous d'avoir Python 3.8+ installé.
\end{alertbox}

\section{Algorithme}

La méthodologie suit le processus itératif décrit ci-après.
