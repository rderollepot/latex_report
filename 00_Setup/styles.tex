% --- 1. MARGES & GÉOMÉTRIE ---
\geometry{
    top=3cm, bottom=2cm, left=2cm, right=2cm,
    headheight=2.5cm, footskip=1.2cm
}

% --- 2. HEADER & FOOTER (FancyHDR) ---
\pagestyle{fancy}
\fancyhf{}

% Header : Logos dynamiques via variables
\fancyhead[C]{
    \fancyhdrbox{\includegraphics[height=0.9cm]{\LogoLeft}}
    \hfill
    \fancyhdrbox{\includegraphics[height=1.2cm]{\LogoCenter}}
    \hfill
    \fancyhdrbox{\includegraphics[height=1.2cm]{\LogoRight}}
}

% Footer
\fancyfoot[L]{\footnotesize \textcolor{gray}{\textit{\ReportTitle}}}
\fancyfoot[R]{\footnotesize \textcolor{gray}{Page \thepage\ / \pageref*{LastPage}}}

% Filets colorés
\renewcommand{\headrulewidth}{0.4pt}
\renewcommand{\headrule}{\hbox to\headwidth{\color{primaryColor}\leaders\hrule height \headrulewidth\hfill}}
\renewcommand{\footrulewidth}{0.4pt}
\renewcommand{\footrule}{\hbox to\headwidth{\color{primaryColor}\leaders\hrule height \footrulewidth\hfill}}

% Force le style fancy sur les chapitres
\assignpagestyle{\chapter}{fancy}

% --- 3. NUMÉROTATION ET TABLE DES MATIÈRES ---

% Profondeur de la numérotation
% 0=Chapitre, 1=Section, 2=Subsec, 3=Subsubsec (On active donc 1.1.1.1)
\setcounter{secnumdepth}{3}

% Profondeur de la Table des Matières
\setcounter{tocdepth}{3}

% Désolidarise les figures et tableaux des chapitres (Fig 1, 2...)
\counterwithout{figure}{chapter}
\counterwithout{table}{chapter}


% --- 4. TITRES (TitleSec) - Configuration Hiérarchique ---

% Chapitre (Niveau 0) - \huge (Très grand)
\titleformat{\chapter}[display]
    {\normalfont\huge\bfseries\color{primaryColor}}
    {\chaptertitlename\ \thechapter}{20pt}{\Huge}

% Section (Niveau 1) - \Large (Grand)
\titleformat{\section}
    {\normalfont\Large\bfseries\color{primaryColor}}
    {\thesection}{1em}{}

% Sous-section (Niveau 2) - \large (Moyen)
\titleformat{\subsection}
    {\normalfont\large\bfseries\color{secondaryColor}}
    {\thesubsection}{1em}{}

% Sous-sous-section (Niveau 3) - \normalsize Gras (Standard renforcé)
\titleformat{\subsubsection}
    {\normalfont\normalsize\bfseries\color{tertiaryColor}}
    {\thesubsubsection}{1em}{}

% Paragraphe (Niveau 4) - \normalsize Italique (Standard distingué)
\titleformat{\paragraph}
    {\normalfont\normalsize\itshape\color{primaryColor}} 
    {\theparagraph}{1em}{}

% Sous-paragraphe (Niveau 5) - \small (Petit - Optionnel)
\titleformat{\subparagraph}
    {\normalfont\small\itshape\color{gray}}
    {\thesubparagraph}{1em}{}


% --- 5. ESPACEMENTS ---
% Format : \titlespacing*{commande}{gauche}{avant}{après}
% Note : Pour les titres non "run-in", le dernier argument est l'espace VERTICAL après le titre.

\titlespacing*{\chapter}{0pt}{50pt}{40pt}
\titlespacing*{\section}{0pt}{3.5ex plus 1ex minus .2ex}{2.3ex plus .2ex}
\titlespacing*{\subsection}{0pt}{3.25ex plus 1ex minus .2ex}{1.5ex plus .2ex}
\titlespacing*{\subsubsection}{0pt}{3.25ex plus 1ex minus .2ex}{1.5ex plus .2ex}

% Pour Paragraph et Subparagraph, on met un espace vertical positif en dernier argument
\titlespacing*{\paragraph}{0pt}{3.25ex plus 1ex minus .2ex}{1em}      
\titlespacing*{\subparagraph}{0pt}{3.25ex plus 1ex minus .2ex}{1em} 
% Note : j'ai mis 0pt à gauche pour subparagraph pour l'aligner, 
% remplacez par \parindent si vous voulez qu'il soit décalé.