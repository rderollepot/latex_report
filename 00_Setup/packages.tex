% =============================================================================
% PACKAGES - Chargement des extensions LaTeX
% =============================================================================
% Ce fichier charge tous les packages nécessaires au document.
% L'ordre de chargement est important pour éviter les conflits.

% =============================================================================
% LANGUE ET TYPOGRAPHIE
% =============================================================================

\usepackage[french]{babel}
\usepackage{mathtools}  % Extension d'amsmath - DOIT ÊTRE AVANT unicode-math

% Polices modernes et complètes (Libertinus)
% Note: libertinus charge fontspec automatiquement avec XeLaTeX/LuaLaTeX
\PassOptionsToPackage{warnings-off={mathtools-colon,mathtools-overbracket}}{unicode-math}
\usepackage{libertinus}  % Police Serif, Sans, Mono et Math

% =============================================================================
% BIBLIOGRAPHIE
% =============================================================================

\usepackage[backend=biber, style=numeric, sorting=none]{biblatex}
\addbibresource{references.bib}

% =============================================================================
% MISE EN PAGE ET GRAPHISME
% =============================================================================

\usepackage{geometry}
\usepackage{graphicx}
\usepackage{xcolor}
\usepackage{titlesec}
\usepackage{fancyhdr}
\usepackage{lastpage}
\usepackage[most]{tcolorbox}  % Encadrés modernes et colorés
\usepackage{float}            % Placement H des figures
\usepackage{caption}          % Meilleure gestion des légendes
\usepackage{subcaption}       % Sous-figures

% =============================================================================
% TABLEAUX ET LISTES
% =============================================================================

\usepackage{tabularx}   % Tableaux à largeur automatique
\usepackage{booktabs}   % Traits horizontaux élégants
\usepackage{enumitem}   % Personnalisation avancée des listes

% =============================================================================
% SCIENCES ET TECHNIQUES
% =============================================================================

% Package siunitx pour les unités physiques
% Indispensable pour une typographie correcte des mesures.
\usepackage{siunitx}
\sisetup{
    locale = FR,             % Conventions typographiques françaises
    detect-all,              % Utilise la police du texte environnant
    per-mode = symbol,       % Affiche "km/h" au lieu de "km h^-1"
    group-separator = {\ },  % Espace comme séparateur de milliers
    inter-unit-product = \., % Point entre les unités (ex: N.m)
    sticky-per               % Évite les ambiguïtés mathématiques
}

% =============================================================================
% UTILITAIRES
% =============================================================================

\usepackage[cache=false]{minted}  % Coloration syntaxique du code
\usemintedstyle{friendly}
\usepackage{csquotes}             % Citations intelligentes
\usepackage{lipsum}               % Texte de remplissage (lorem ipsum)
\usepackage[all]{nowidow}         % Évite les veuves et orphelines
\usepackage{microtype}            % Améliore la justification

% =============================================================================
% NAVIGATION ET HYPERLIENS
% =============================================================================
% Packages de navigation à charger en dernier pour éviter les conflits.

\usepackage[colorlinks=true, linkcolor=linkCol, urlcolor=linkCol, citecolor=linkCol]{hyperref}
\usepackage{cleveref}  % Références intelligentes (\cref)

% Configuration du chemin vers les images
\graphicspath{{img/}}